
This chapter will contain all the relevant theory of the Distributed Hypothesis Testing (DHT) model, not including graphs or sociology. The model is applied on a network or graph of nodes, which are able to interact. First, I will introduce the developments of the model since its inception, then I will define it as it has been used for this Master thesis (with relevant constraints). Lastly I will present the algorithms which have been used and any potential differences from the project thesis preceding this Master thesis.

This chapter is inspired by my project thesis "Beliefs and the Propagation of Misinformation in Social Networks" at NTNU in 2025 \cite{project_thesis} (specifically section 3.2), Diana Riazi's papers and PhD thesis \cites{RiaziPublicAndPrivateBeliefsUnderDisinformation}{riazi2024fightspreadfakenews}{RiaziMitigatingDisinformation}{riazi2025phd}, and the paper on the DHT model by Anusha Lalitha et al. \cite{Social_Learning_and_Distributed_Hypothesis_Testing}. 

\section{History}\label{sec:DHT_history}

This section contains a brief history of developments of the DHT model from its inception by Robert Tenney and Nils Sandell Jr. in 1981\cite{tenney1981detectionwithdistributedsensors} until its formulation by Lalitha et al. in 2014\cite{Social_Learning_and_Distributed_Hypothesis_Testing}. 

The earliest work done on the subject was motivated by the fact that in a distributed sensor network, it can be costly or inefficient to make all sensors communicate with a central hub. Thus, the sensors must be able to process signals themselves using an optimal decision rule. Tenney et al. created a model based on the formulation of decentralized hypothesis testing \cite{tenney1981detectionwithdistributedsensors}. They found that the optimal decision rule is that the sensors perform likelihood ratio tests on the statistically independent observations they make. 

Over the following decades, research further developing the model was mostly concerned on distributed sensor networks. By minimizing loss, Firooz determined the structure of the optimal decision rule given an arbitrary number of sensors and hypothesis \cite{4104192}. Viswanathan et al. made a review study of the entire filed in 1997 \cite{554208}. In 2004 Alanyali et al. applied belief propagation, and thus interaction between sensors, to make the network reach a consensus concerning an observation \cite{1384706}. Halme et al. included spatial dependency in the model \cite{8755597}. 

In 2014 (and revised in 2018), Lalitha et al. published "Social Learning and Distributed Hypothesis Testing" \cite{Social_Learning_and_Distributed_Hypothesis_Testing}. This paper included social learning in the DHT model, implying that the nodes in the network interact in such a way that they can learn from each other. Lalitha et al. introduced the notion of public and private beliefs, such that interactions between nodes happened through public beliefs. These were formed by a bayesian update of private beliefs using likelihood functions made from observations. This formulation of the model was inteded to be used on directed graphs.

In 2024 and 2025, Diana Riazi published three papers with her PhD thesis using the DHT model to simulate misinformation propagation in social networks \cites{RiaziPublicAndPrivateBeliefsUnderDisinformation}{riazi2024fightspreadfakenews}{RiaziMitigatingDisinformation}{riazi2025phd}. This is the first and, so far, only study using the DHT model to simulate propagation of beliefs in a network consisting of humans. The mathematical formulation will be presented in section \ref{}.

\section{Differences from project thesis}